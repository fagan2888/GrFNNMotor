\documentclass{article}
%%%%%%%%%%%%%%%%%%%%%%%%%%%%%%%%%%%%%%%%%%%%%%%%%%%%%%%%%%%%%%%%%%%%%%%%%%%%%%%%%%%%%%%%%%%%%%%%%%%%%%%%%%%%%%%%%%%%%%%%%%%%
\usepackage[pdftex]{graphicx}
%\usepackage{amsmath}



\begin{document}


\section{A note for Svar function in Matlab}

\noindent
\begin{verbatim}
 Marco Aiolfi
 maiolfi@iol.it
 Bocconi University and Banca Intesa, Milan
\end{verbatim}


SVAR verifies the identification conditions for a given structural form to
be imposed on an estimated VAR model. The require inputs are the set of
constraints to be placed on the elements of the A and B matrices so that
\begin{eqnarray}
Vec(A) &=&s_{A}\cdot \gamma _{A}+d_{A}  \label{res} \\
Vec(B) &=&s_{B}\cdot \gamma _{B}+d_{B}.  \nonumber
\end{eqnarray}

Here is an example.

Consider a 2 variables VAR and suppose you want Choleski identification:
\begin{equation}
A=\left[
\begin{array}{cc}
1 & 0 \\
a_{21} & 1
\end{array}
\right] ,B=\left[
\begin{array}{cc}
b_{11} & 0 \\
0 & b_{22}
\end{array}
\right] .
\end{equation}

Using \ref{res} the constraints can be rewritten as follows:
\begin{eqnarray}
\left[
\begin{array}{c}
1 \\
a_{21} \\
0 \\
1
\end{array}
\right]  &=&\left[
\begin{array}{c}
0 \\
1 \\
0 \\
0
\end{array}
\right] \cdot \left[ a_{21}\right] +\left[
\begin{array}{c}
1 \\
0 \\
0 \\
1
\end{array}
\right]  \\
\left[
\begin{array}{c}
b_{11} \\
0 \\
0 \\
b_{22}
\end{array}
\right]  &=&\left[
\begin{array}{cc}
1 & 0 \\
0 & 0 \\
0 & 0 \\
0 & 1
\end{array}
\right] \cdot \left[
\begin{array}{c}
b_{11} \\
b_{22}
\end{array}
\right] +\left[
\begin{array}{c}
0 \\
0 \\
0 \\
0
\end{array}
\right] .
\end{eqnarray}
After creating $s_{A},d_{A},s_{B},d_{B}$ the SVAR procedure is run with
\begin{verbatim}
[a,b,a_se,b_se]=svar1(results,sa,sb,da,db,afree,bfree)
\end{verbatim}

\noindent
where \texttt{results} is a VARE structure, \texttt{sa},\texttt{sb},\texttt{da%
},\texttt{db}, are the constraints and \texttt{afree} and \texttt{bfree} are
the number of free parameters in $A$ and $B$ matrices.

The demo file \textit{svar\_d.m} gives an example of SVAR procedure for a 5
variables VAR.

The interested reader can find details about identification and estimation
of a structural VAR models in Amisano and Giannini (1997). A good reference
for alternative identification schemes and their application to monetary
policy analys is Favero (2000).

\section{References}
\begin{enumerate}
\item[]   Amisano, G., and Giannini, C., (1997). \textit{Topics in
structural var econometrics}. Second edition, Springer Verlag, New York.

\item[] Favero, C. A., (2000). Applied Macroeconometrics. Oxford
University Press, Oxford.

\end{enumerate}

\end{document}
